\documentclass[12pt, a4paper, oneside]{ctexbook}
\usepackage{amsmath, indentfirst, amsthm, amssymb, bm, graphicx, hyperref, mathrsfs}
\usepackage{amssymb}
\usepackage{stmaryrd}

\title{{\Huge{\textbf{ODE}}}}
\author{Hao Y.H}
\date{\today}
\linespread{1.5}
\setlength{\parindent}{2em}
\newtheorem{theorem}{定理}[section]
\newtheorem{definition}[theorem]{定义}
\newtheorem{lemma}[theorem]{引理}
\newtheorem{corollary}[theorem]{推论}
\newtheorem{example}[theorem]{例}
\newtheorem{proposition}[theorem]{命题}

\newcommand*{\dif}{\mathop{}\!\mathrm{d}}
\begin{document}

\maketitle

\pagenumbering{roman}
\setcounter{page}{1}

\begin{center}
    \Huge\textbf{前言}
\end{center}~\

低廉而有效的快乐.
~\\
\begin{flushright}
    \begin{tabular}{c}
        Hao Y.H\\
        \today
    \end{tabular}
\end{flushright}

\newpage
\pagenumbering{Roman}
\setcounter{page}{1}
\tableofcontents
\newpage
\setcounter{page}{1}
\pagenumbering{arabic}

\chapter{ODE初级解法}

summary.

    \section{conception}
        常微分方程解决的是求函数的问题,其中,未知函数的自变量唯一.\par
        首先约定术语如下:
        \begin{definition}[阶]
            未知函数的导数的最高阶数即为DE的阶.一般的n阶ODE可表示为:
            \begin{align}
                F(t,x,\frac{df}{dt},\cdots,\frac{df^n}{dt^n})=0
            \end{align}
        \end{definition}
        \begin{definition}[解与定义空间]
            若函数$\phi(x)$在某区间$[a,b]$内有$n$阶连续导数,且将函数$x= \phi(t)$代入方程(1.1)后,
            可使得等式
            \begin{align*}
                F(t,\phi(t),\phi^{'}(t),\cdots,\phi^{(n)} (t)) = 0
            \end{align*}
            在$[a,b]$中恒成立,称函数$ x= \phi(t) $为方程(1.1)的解,称$[a,b]$为解的定义空间.\par
            当$x= \phi(t)$不易求得而$ \phi(t,x)=0 $易于求得时,后者确定的隐函数为方程(1.1)的解,
            则称$ \phi(t,x) = 0 $为方程(1)的积分.\par
            对于一个微分方程,求其积分,相当于求得其解.

        \end{definition}
        \begin{definition}[积分曲线]
            解在$t,x$平面上的几何表示---平面曲线,称为方程(1.1)的积分曲线.
        \end{definition}
        \begin{definition}[方向场---微分方程的几何解释]
            当一阶ODE已解,总能以$t,x$表示出积分曲线上任一一点的斜率,因此可依据积分曲线作出有向线段,即方向场.\par
            欧拉折线以方向场为原理.
        \end{definition}
        \begin{definition}[变系数线性微分方程]
        \end{definition}
    \section{一阶方程的初等解法}
        \subsection{分离变量法}
        \begin{definition}[变量可分离方程]
            \begin{align}
                \frac{dx}{dt} = f(x)\cdot g(t)
            \end{align}
        \end{definition}
        \begin{definition}[耦合可分离方程-齐次方程]
            \begin{align}
                \frac{dx}{dt}=g(\frac{x}{t})
            \end{align}
            \begin{align}
                \frac{dx}{dy}=\frac{a_1x+b_1y+c_1}{a_2x+b_2y+c_2}
            \end{align}
            对于方程(1.4),试作变换:
            $\left\{
            \begin{aligned}
                x &= \xi + h\\
                y &= \eta + k
            \end{aligned}
            \right$\par
            令变换后的分子分母的常数项等于零,得到$h,k$,当线性方程组行列式为0时,ODE退化,其解是trivial的,于此不作赘述.
        \end{definition}
        \begin{definition}[线性方程]
        \end{definition}
        \begin{example}[因果变量互易一例]
            $\frac{dx}{dt}(x^3+\frac{t}{x})=1$
        \end{example}
        \begin{definition}[全微分方程和积分因子]
            \begin{align}
                P(x,y)dx+Q(x,y)dy=0
            \end{align}
            若满足柯西黎曼方程,则称全微分方程(1.5)是恰当的.若存在形如(1.5)的方程不满足柯西黎曼方程,然而乘以某适当函数后,满足柯西黎曼方程,称此函数为积分因子
            \begin{align}
            \mu(x,y)[P(x,y)dx+Q(x,y)dy]=0
            \end{align}
            对于(1.6),$\mu(x,y)$的求解通过柯西黎曼方程实现,可令$\mu(x,y)$关于x或y的偏导等于0,从而简化方程求解难度.
        \end{definition}
    \section{导数未解出的一阶方程}
        本节研究的方程的一般形状为:
        \begin{align}
            F(t,x,x^{'})=0
        \end{align}
        对于(1.7),三自变量,可能得到三种隐函数,本节研究导数未解出的一阶方程.
        \begin{definition}[方程$x = g(t,x^{'})$与$t = h(x,x^{'})$]
        \end{definition}
        1.对于方程$x = g(t,x^{'})$:\par
        令$p=\frac{\dif x}{\dif t}$,方程取t导数,得到:
        \begin{align}
            p = \frac{\partial g(t,p)}{\partial t}+\frac{\partial g(t,p)}{\partial p}\cdot \frac{\dif p}{\dif t}
        \end{align}
        2.对于方程$t = h(x,x^{'})$:
        令$\frac{1}{p} = \frac{dt}{dx}$即可求解.
        \begin{example}
            \begin{align}
                x(\frac{dx}{dt})^2+-2t\frac{dx}{dt}+x=0
            \end{align}\par
            Therefore, $t =\frac{x}{2p}+\frac{xp}{2}$.\par
            Derivative of x, $\frac{1}{p} = \frac{1}{2p}+\frac p2+ (\frac x2-\frac{x}{2p^2})\frac{dp}{dx}$.\par
            Multiply the both sides of the equation by $2p^2$:$(p^2-1)(x\frac{dp}{dx}+p)=0$\par
            The rest of part is obviously trivial.
        \end{example}
        \begin{example}
            clairaut equation,where $f(u)$ is continuously derivable, and $ f^{'}(u)\neq constant$:
            \begin{align}
                x &= t\cdot\frac{dx}{dt}+f(\frac{dx}{dt})
            \end{align}\par
            which is equal to: $ x = tp+f(p)$.\par
            Derivative of t, $ p = p + (t+f^{'}(p))\frac{dp}{dt} $.\par
            Trivial.
        \end{example}
    \section{微分方程组的初等积分法与首次积分}
        \subsection{转化为高阶方程}
        \begin{example}
            对于微分方程组:\par
            \left\{
            \begin{aligned}
                \frac{dx}{dt}&= y\\
                \frac{dy}{dt}&= x
            \end{aligned}
            \right.\par
            一式求导,代入二式即可.
        \end{example}
        \subsection{首次积分法}
        若能够通过一定的运算得到易于积分的全微分方程,可得到该微分方程的原函数,称此原函数为首次积分。
        \begin{example}[对称性一例]
            \begin{align}
                \frac{dx}{cy-bz} = \frac{dy}{az-cx}=\frac{dz}{bx-ay}
            \end{align}
            分母求和为0,则:
            \begin{align}
                \frac{xdx}{x(cy-bz)} = \frac{ydy}{y(az-cx)}=\frac{zdz}{z(bx-ay)} = \frac{xdx+ydy+zdz}{0}\\
                \frac{adx}{a(cy-bz)} = \frac{bdy}{b(az-cx)}=\frac{cdz}{c(bx-ay)} = \frac{adx+bdy+cdz}{0}
            \end{align}
            为使等式成立,则$xdx+ydy+zdz=0,adx+bdy+cdz=0$
            积分可得:
            \left\{
            \begin{aligned}
                x^2+y^2+z^2&=C_1\\
                ax+by+cz&= C_2
            \end{aligned}
            \right
        \end{example}
\chapter{常系数线性微分方程}
        \section{二阶常系数线性微分方程的求解}
        特征根即可
        \section{二阶非齐次微分方程——常数变易法}
        与其寻找常数变易法的因果逻辑,纠结此刻的因果,不如承认常数变易法是强关联的,因为充要性都是验证过的,强关联是毋庸置疑的。
        二阶非齐次常系数线性方程:
        \begin{align}
            a_0\frac{d^2x}{dt^2}+a_1\frac{dx}{dt}+a_{2}x=f(t)
        \end{align}
        其不同的2个特征根满足:
        \begin{align}
            a_0\lambda^2+a_1\lambda+a_{2}=a_0(\lambda-\lambda_1)(\lambda-\lambda_2)=0
        \end{align}
        2特征根使得特征方程等于0,需承认此时特征根对应的解亦然是原方程的解.\par
        则对于方程(2.1)有:
        \begin{align}
            a_{0}\left\{\frac{d^{2}x}{dt^{2}}-(\lambda_{1}+\lambda_{2})\frac{dx}{dt}+\lambda_{1}\lambda_{2}x\right\}=0,
        \end{align}
        组合:
        \begin{align}
            a_{0}\left\{\frac{d}{dt}\Big(\frac{dx}{dt}-\lambda_{1}x\Big)-\lambda_{2}\Big(\frac{dx}{dt}-\lambda_{1}x\Big)\right\}=\mathbf{0}.
        \end{align}
        最后得到:
        \begin{align}
            a_{0}\left\{\frac{dy}{dt}-\lambda_{2}y\right\}&=f\left(t\right),\\\frac{d}{dt}\left\{e^{-\lambda_{1}t}y\right\}&=\frac{1}{a_{0}}f\left(t\right)e^{-\lambda_{2}t},
        \end{align}
        二特征根组合,得到最终解为:
        \begin{align}
            x=c_{1}e^{\lambda_{1}t}+c_{2}e^{\lambda_{1}t}+\frac{1}{a_{0}}\int_{0}^{t}\frac{e^{\lambda_{1}(t-s)}-e^{\lambda_{1}(t-s)}}{\lambda_{1}-\lambda_{2}}f(s)ds
        \end{align}\par
        此即为常数变易法.
\chapter{线性常微分方程组}
微分方程与矩阵的关联:remain to be done
    \section{矩阵值函数与向量值}
        本节研究的矩阵如下,当行列数有为1时,退化为向量值矩阵.
        \[
            \begin{pmatrix}
            a_{11}(t) & a_{12}(t) & \dots & a_{1n}(t) \\
            a_{21}(t) & a_{22}(t) & \dots & a_{2n}(t) \\
            \vdots & \vdots & \ddots & \vdots \\
            a_{n1}(t) & a_{n2}(t) & \dots & a_{nn}(t)
            \end{pmatrix}
        \]
        \begin{definition}[矩阵值函数的微分与积分]
            定义矩阵值函数的微积分如下:
            \begin{align}
            \frac{dA(t)}{dt}&=\frac{d(a_{ij})}{dt}_{i=1,2,3,\cdots,n;j = 1,2,3,\cdots,m.}\\
            \int^t_{t_0}A(x)dx&=(\int^t_{t_0}a_{ij}(x)dx)_{i=1,2,3,\cdots,n;j=1,2,3,\cdots,m.}
            \end{align}
            即矩阵内对应元素求微分或求微分即可.
        \end{definition}
        \begin{proposition}[矩阵值函数微积分的性质选]
            1.$ \frac{dA^{-1}(t)}{dt}=-A^{-1 }(t)\cdot\frac{dA(t)}{dt}A^{-1}(t) $ \par
            其原因是矩阵左右乘的不相等.
        \end{proposition}
        \subsection{矩阵与向量的范数}
            \begin{definition}
                假设$A=(a_{ij})$与$B=(b_{ij})$为2个$n\times m$阶矩阵,定义其内积为:\\
                \centering $<A,B>=\sum^n_{i=1}\sum_{j=1}^{m}a_{ij}b_{ij} $\\
                矩阵A的范数为:\\
                \centering $\|A\|=\sqrt{\sum_{i=1}^{n}\sum_{j=1}^{m}(a_{ij})^2 }$
            \end{definition}
            \begin{proposition}
                \begin{itemize}
                    关于范数与内积:\par
                    \item[1.]正定性:$\|A\|\geq 0$,当且仅当$A=0$时,$\|A\|=0$
                    \item[2.]齐次性:$\|\lambda A\|=|\lambda |\|A\|$
                    \item[3.]三角不等式:$\|A+B\|\leq\|A\|+\|B\|$
                    \item[4.]$|\langle A,B\rangle|\leq\|A\|\cdot\|B\|$
                    \item[5.]若$A$为$n\times m$阶矩阵,$B$为$m\times l$阶矩阵,则: $\|AB\|\leq\|A\|\cdot\|B\|$
                    \item[6.]$\left\|\int_{a}^{\beta}A(t)dt\right\|\leq \int_{a}^{\beta}\|A(t)\|dt$
                \end{itemize}
            \end{proposition}
            \begin{definition}[矩阵序列的极限]
                设$A_1,A_2,A_3,\cdots,A_k,\cdots$是一系列的$n\times m$阶矩阵,如果存在$A$,使得当$k \rightarrow+\infty$时,有:\\
                \centering $\|A_k-A\|\rightarrow 0$\\
                \raggedright 则称$A$为$\{A_k\}$的极限.其数学表示和一般极限是一致的.
            \end{definition}
        \subsection{线性微分方程组的矩阵表示}
        本节研究的线性常微分方程组是:
            \begin{equation}
                \begin{cases}
                    \frac{dx_1}{dt}=a_{11}(t)x_1+a_{12}(t)x_2+\cdots+a_{1n}(t)x_n+f_1(t),\\
                    \frac{dx_2}{dt}=a_{21}(t)x_1+a_{22}(t)x_2+\cdots+a_{2n}(t)x_n+f_2(t),\\
                    \frac{dx_3}{dt}=a_{31}(t)x_1+a_{32}(t)x_2+\cdots+a_{3n}(t)x_n+f_3(t),\\
                    \cdots\\
                    \frac{dx_n}{dt}=a_{n1}(t)x_1+a_{n2}(t)x_2+\cdots+a_{nn}(t)x_n+f_n(t),
                \end{cases}\label{eq:Linear ODE}
            \end{equation}
        用矩阵表示为:
            \begin{align}
                \frac{d\mathbf{\mathit{x}}}{dt}=\mathbf{\mathit{A}} (t)x+\mathbf{\mathit{f}}(t).
            \end{align}
        \subsection{$e^{\mathbf{\mathit{A}}t}$}
        \begin{definition}[$e^{\mathbf{{\mathit{A}}}t}$]
            设$A$为$n$阶矩阵,则定义$e^{\mathbf{{\mathit{A}}}t}$为:
            \begin{align}
                e^{\mathbf{{\mathit{A}}}t}=E+At+\frac{1}{2!}(At)^2+\cdots+\frac{1}{k!}(At)^k+\cdots.
            \end{align}
        \begin{theorem}[非齐次微分方程组解]
            $f(t)$在$(\alpha,\beta)$是continuous的,则非齐次微分方程组:
            \begin{align}
                \frac{d\mathbf{\mathit{x}}}{dt}=A\mathbf{\mathit{x}}+\mathbf{\mathit{f}}(t)
            \end{align}
            满足初值条件$\mathbf{\mathit{x}}(t_0)=\mathbf{\mathit{x_0}}$的解在$(\alpha,\beta)$是唯一的,且:
            \begin{align}
                \mathbf{\mathit{x(t)}}=e^{A(t-t_0)}\mathbf{\mathit{x_0}}+\int_{t_0}^{t}e^{A(t-s)}\mathbf{\mathit{f}}(s)ds
            \end{align}
        \end{theorem}
            
        \end{definition}
    \section{常系数微分方程组的求解}
    本节给出了特征值特征向量与常系数微分方程的关系.对于忽视代数重数与几何重数问题的学生(我),有重根情况下的求解是新颖的.
    此处纪录一矩阵:\par \centering
    \begin{pmatrix}
        1&1&0\\
        -1&3&0\\
        2&-6&0
    \end{pmatrix}\par \raggedright
    同时建议注意参数的位置.对于一个特征值而言,解得的x, y有相同的参数.这就出现了一个问题:需要将所有的特征值与特征向量均求解出来,而出现重根时,是无法完成的(或者说,有时重根无法求得所有的向量,导致张不成微分方程组的维度,而此时的微分方程组的解必然有多个'线性无关'的基,这是矛盾存在之处)
    \begin{example}[齐次无重根一例]
        求解线性微分方程组的解:\\
        \centering
        \begin{aligned}
            \frac{dx}{dt}&=2x+y\\
            \frac{dy}{dt}&=y
        \end{aligned}\par
        \raggedright
        线性微分方程组的矩阵A:\\
        \centering
        \begin{pmatrix}
            2&1\\
            0&1
        \end{pmatrix}\par
        \raggedright
        其特征值为1,2.
        对于$ \lambda_1=1$,其特征向量满足关系:\par
        \centering
        \begin{aligned}
            (A-\lambda_1E)X=\begin{pmatrix}
                1&1\\
                0&0
            \end{pmatrix}\cdot\begin{pmatrix}
                \phantom{0}x_1\phantom{0}\\
                \phantom{0}x_2\phantom{0}
            \end{pmatrix}=\begin{pmatrix}
                0\\0
            \end{pmatrix}
        \end{aligned}\par
        \raggedright
        所以$ x=-c_{1}e^t,y = c_{1}e^t$是方程组的一个解.另一特征值对应的解不再赘述.
        最终方程组的解为:\par
        \centering
        \begin{aligned}
            x &= -c_{1}e^t+c_{2}e^{2t}\\
            y &= +c_{1}e^t
        \end{aligned}\par
        \raggedright
    \end{example}
    从上述求解看出,无重根时,特征值代表指数系数大小,特征向量代表系数的权重.\\
    \begin{definition}[一般解的形式]
        对于一阶常系数线性微分方程组
        \begin{align}
            \frac{d\mathbf{\mathit{x}}}{dt}=A\mathbf{\mathit{x}}
        \end{align}
        A的n个特征值为$\lambda_1,\lambda_2,\cdots,\lambda_{n}$,则存在n个循环向量系数多项式
        $ p_1(t),p_2(t),\cdots,p_n(t)$,使得$ p_1(0),p_2(0),\cdots,p_n(0)$线性无关.
        且方程组解为:
        \begin{align}
            x&=\sum_{j=1}^nc_jp_j(t)e^{\lambda_jt}\\
            \boldsymbol{p}\left(t\right)&=\boldsymbol{a}_{0}\frac{t^{k}}{k!}+\boldsymbol{a}_{1}\frac{t^{k-1}}{\left(k-1\right)!}+\cdots+\boldsymbol{a}_{k}\\
            A\boldsymbol{a}_0&= \lambda\boldsymbol{a}_0,\nonumber  \\
            A\boldsymbol{a}_1&= \lambda\boldsymbol{a}_1+\boldsymbol{a}_0,\nonumber  \\
            \cdots \\
            A\boldsymbol{a}_k&=\lambda\boldsymbol{a}_n+\boldsymbol{a}_{k-1}.\nonumber
        \end{align}
    \end{definition}
    \begin{example}[有重根一例]
        求常微分方程组\\
        \centering
        \begin{cases}
            &\frac{d{x}_1}{d{t}} =-4x_1+x_2+3x_3+2x_4,  \\
            &\frac{d{x}_2}{d{t}} =-2x_1-x_2+2x_3,  \\
            &\frac{d{x}_3}{dt} =-3x_1+x_2+2x_3+2x_4  \\
            &\frac{d{x}_4}{dt} =x_1-x_8-x_4
        \end{cases}\par
        \raggedright
        方程组特征多项式为\\
        \centering
        \begin{aligned}
            |A-\lambda E|=\begin{vmatrix}-4-\lambda&1&3&2\\-2&-1-\lambda&2&0\\-3&1&2-\lambda&2\\1&0&-1&-1-\lambda\end{vmatrix}.
        \end{aligned}\par
        \raggedright
        特征值为-1,4重根,几何重数为2,特征向量为:\\
        \centering
        \begin{pmatrix}
            1\\
            0\\
            1\\
            0\\
        \end{pmatrix}
        与\begin{pmatrix}
            0\\
            -2\\
            0\\
            1\\
        \end{pmatrix}\par
        \raggedright
        值得注意的是,考虑到化简过程中存在倍乘操作,非齐次情况下失效,所以特征向量应代入原始矩阵当中.
        对于第一个特征向量而言,其循环向量为:\par\centering
        \begin{aligned}
            a_1 = k_1\begin{pmatrix}
                1\\
                0\\
                1\\
                0\\
            \end{pmatrix}+k_2\begin{pmatrix}
                0\\-2\\0\\1\\
            \end{pmatrix}+\begin{pmatrix}
                0\\1\\0\\0\\
            \end{pmatrix}
        \end{aligned}\par\raggedright
        所以,\begin{aligned}
            p_1 (t)=a_{0}t+a_1 = \begin{pmatrix}
                t\\1\\t\\0\\
            \end{pmatrix}
        \end{aligned}\par
        最终,解为:\begin{aligned}
            \boldsymbol{x}=\left(c_1\boldsymbol{p}_1(t)+c_2\frac{d\boldsymbol{p}_1(t)}{dt}+c_3\boldsymbol{p}_2(t)+c_4\frac{d\boldsymbol{p}_2(t)}{dt}\right)e^{-t}
        \end{aligned}\par
        解的形式说明,得到循环向量后可直接作求微分(或许可以直到不出现t为止,perhaps),得到最终答案.

    \end{example}
    \section{初值问题解的存在唯一性}
    本节研究变系数矩阵下的解存在唯一性问题.其大抵等价于,当内部环境参数改变后,系统的稳定性问题.
    \section{解的结构}
    微分方程组的结构与线性方程组的结构类似.
    \subsection{刘维尔公式}
    \begin{definition}[解方阵,朗斯基行列式]
        考虑$R^{n}$上的一阶线性微分方程$\frac{dx}{dt}=A(t)x$
        设$x_1(t),x_2(t),x_3(t),\cdots,x_n(t)$为基本解组,
        以基本解组为列向量的方阵$ \Phi(t) $,即为基本解方阵,其行列式即为朗斯基行列式\\
        满足$\frac{d\Phi(t)}{dt}=A(t)\Phi(t)$\\
        朗斯基行列式满足:$\frac{dW(t)}{dt}\equiv\sum_{k=1}^na_{kk}(t)W(t)=\text{tr}A(t)W$

    \end{definition}
    不采用书中的内容:现在根据二阶齐次线性微分方程说明:若给出其中一个特解,则另一个线性无关的特解是唯一确定的.
    \begin{aligned}
        y_{2}(x)=y_{1}(x)\int\frac{e^{-\int P(x)dx}}{y_{1}^{2}(x)}dx
    \end{aligned}
    \subsection{常数变易公式}
    齐次线性方程组的解为:$\mathbf{\mathit{x}} = \phi(t)\mathbf{\mathit{c}}$,合理联想非齐次解为:
    \begin{align}
        \mathbf{\mathit{x}} = \phi(t)\mathbf{\mathit{c}}(t)
    \end{align}
    derivative of t,运用解方阵的性质:
    \par \centering
    \begin{aligned}
        \frac{d\boldsymbol{x}}{dt}&=\frac{d\boldsymbol{\Phi}(t)}{dt}\boldsymbol{c}(t)+\boldsymbol{\Phi}(t)\frac{d\boldsymbol{c}(t)}{dt}\\&=\boldsymbol{A}(t)\boldsymbol{\Phi}(t)\boldsymbol{c}(t)+\boldsymbol{\Phi}(t)\frac{d\boldsymbol{c}}{dt}
    \end{aligned}\par \raggedright
    由(3.12)知:$\frac{d\boldsymbol{x}}{dt}\equiv A(t)\boldsymbol{x}(t)+\boldsymbol{f}(t)\equiv\boldsymbol{A}(t)\boldsymbol{\Phi}(t)\boldsymbol{c}(t)+\boldsymbol{f}(t)$\\
    比较系数,得到:$\boldsymbol{\Phi}(t)\frac{d\boldsymbol{c}}{dt}\boldsymbol{=}\boldsymbol{f}(t).$\\
    则最终解为:
    \begin{align}
        \boldsymbol{x}=\boldsymbol{\Phi}(t)\boldsymbol{c}_0+\int_{t_0}^t\boldsymbol{\Phi}(t)\boldsymbol{\Phi}^{-1}(\boldsymbol{s})\boldsymbol{f}(\boldsymbol{s})d\boldsymbol{s}
    \end{align}
    称$U(t,s)\equiv\Phi(t)\Phi^{-1}(s)$为转移矩阵.
    \begin{example}[非齐次微分方程组,常数变易法一则]
        求解微分方程组:\par\centering
        \begin{cases}
            \frac{dx}{dt}&=2x+y-e^{-t}\\
            \frac{dy}{dt}&=4x+3y+4e^{-t}
        \end{cases}\par\raggedright
        设解为:\par\centering
        \begin{cases}
            x&=c_{1}e^{5t}+c_{2}e^{-t}\\
            y&=2c_{1}e^{5t}-c_{2}e^{-t}
        \end{cases}\par\raggedright
        将解取微分,全部代入题干方程组,得到(原书此处十分shabi,自己可以看看)(此处似乎存在规律,例如等式右边与非齐次项的关系,以及等式左边与求导项的关系):\\
        \par\centering
        \begin{cases}
            e^{5t}\frac{dc_1}{dt}+e^{-t}\frac{dc_2}{dt}&=-e^{-t}\\
            2e^{5t}\frac{dc_1}{dt}-e^{-t}\frac{dc_2}{dt}&=4e^{-t}
        \end{cases}\par\raggedright
        得到了线性方程组,易解得$c_1,c_2$,代入到假设解即可.
    \end{example}\par\raggedright
    \section{二阶变系数线性微分方程}
    本节讨论二阶变系数线性微分方程:
    \begin{align}
        \frac{dx^2}{d^2t}+p(t)\frac{dx}{dt}+q(t)x=f(t)
    \end{align}
    令$y=\frac{dx}{dt},$ (3.14)化作一阶线性微分方程组:\par\centering
    \begin{cases}
        \frac{dx}{dt}&=y\\
        \frac{dy}{dt}&=-q(t)x-p(t)y+f(t)
    \end{cases}\par\raggedright
    或
    \begin{align}
        \frac{d}{dt}\begin{pmatrix}
            x\\y
        \end{pmatrix}=\begin{pmatrix}
            0&1\\-q(t)&-p(t)
        \end{pmatrix}\cdot\begin{pmatrix}
            x\\y
        \end{pmatrix}+\begin{pmatrix}
            0\\f(t)
        \end{pmatrix}
    \end{align}
    \begin{theorem}[二阶微分方程通解]
        对于(3.14)的齐次形式而言,存在2线性无关的解.其通解为:\par\centering
        $ x=c_1\phi_1(t)+c_2\phi_2(t)$
        \par\raggedright
        二阶方阵函数$U(t,s)$是(3.15)的转移矩阵,即:
        \begin{align}
            U(t,s)=\begin{pmatrix}
                \phi_1(t,s)&\phi_2(t,s)\\\phi_1^{'}(t,s)&\phi_2^{'}(t,s)
            \end{pmatrix}
        \end{align}
        (3.16)满足的条件是,其为单位矩阵.值得注意的是,$\phi(t,s)$是方程的解,由(3.13)可看出来.
        由(3.13),微分方程(3.14)的解为:
        \begin{align}
            x=\phi_1(t,t_0)x_0+\phi_2(t,t_0)x_0+\int_{t_0}^t\phi_2(t,s)f(s)ds
        \end{align}
    \end{theorem}
    \begin{theorem}
        如果$k(t)$为二阶常系数线性方程
        \begin{align}
            \frac{dx^2}{d^2t}+p\frac{dx}{dt}+qx=0
        \end{align}
        满足条件:通解初值等于0,通解导函数初值为1.则$x=k(t-s)$也是其解.\\
        若$x=k(t-s)\mid_{t=0}=0,k^{\prime}(t-s)\mid_{t=0}=1$.则\\
        $x=\int_{0}^{t}\boldsymbol{k}\left(t-\boldsymbol{s}\right)\boldsymbol{f}\left(t\right)\boldsymbol{d}\mathbf{s}$
        为二阶非齐次线性微分方程的解.
    \end{theorem}
    \begin{example}[根据齐次求非齐次(疑惑)]
        求解二阶微分方程:
        \begin{align}
            t^2\frac{dx^2}{d^2t}-2x=t
        \end{align}
        当$t>0$时,$x=t^2,x=\frac1t$为齐次解,根据通解原函数值为0,通解导函数值为1,求解转移函数.代入公式(3.17)即可得到最终解.注意:二阶项系数为1,\\\centering
        \begin{aligned}
            x=c_1t^2+\frac{c_2}t+\int_1^t\biggl(\frac{t^2}{3s}-\frac{s^2}{3t}\biggr)\frac1sds
        \end{aligned}
        \par\raggedright
        值得注意的是,获得一个非零解,根据刘维尔公式可得到另一解.\\
        然而,二阶齐次微分方程的解无法通过初等函数以及有限次积分求解,刘维尔已经证明.
    \end{example}
    \begin{theorem}[幂级数求解二阶微分方程(1)]
        若$p(t),q(t)$在收敛域中可展开为幂级数:\par\centering
        \begin{aligned}
            p(t)=\sum_{k=0}^{\infty}p_{k}t^k,q(t)=\sum_{k=0}^{\infty}q_{k}t_k,
        \end{aligned}\par\centering
        则二阶线性微分方程的解在相同的收敛域中也可以展开为收敛的幂级数.\par\centering
        \begin{aligned}
            x = \sum_{k=0}^{\infty}c_{k}t^k
        \end{aligned}\par\centering
    \end{theorem}
    \begin{theorem}[幂级数求解二阶微分方程(2)]
        设$ a(t),b(t) $在收敛域可以展开,则对于二阶线性微分方程
        \begin{align}
            t^2\frac{d^2x}{dt^2}+ta(t)\frac{dx}{dt}+b(t)x=0
        \end{align}
        至少有一个解形如:\\ \centering
        \begin{aligned}
            x=t^{\rho}\sum_{k=0}^{\infty}c_{k}t^k
        \end{aligned}\\ \raggedright
        其中$ \rho$为某常数.
    \end{theorem}
    \begin{example}
        求解二阶线性方程:$ \frac{d^2x}{dt^2}-tx=0 $\\
        设通解为幂级数,代入方程比较系数即可.
    \end{example}
    \begin{example}[Bessel]
        \begin{align}
            t^2\frac{d^2x}{dt^2}+t\frac{dx}{dt}+(t^2-m^2)x=0
        \end{align}
    \end{example}
    \section{二阶线性微分方程的边值问题}
    以往解决的均为初值问题;本节的现实意义为:两点之间的最速降线问题.本文将解决,1.何时存在唯一解.2.如何表示其解.
    \begin{definition}[边值问题形式]
        对于二阶线性微分方程
        \begin{align}
            &\frac{d^2y}{dx^2}+p(x)\frac{dy}{dx}+q(x)y=f(x)\\
            &y(a)=A,y(b)=B
        \end{align}
        假设$p(x),q(x),f(x) $在闭区间$[a,b]$continuous,称其为两点边值问题,(3.23)为边值条件,若二者均为0,称之为齐次条件,反之非齐次条件.
    \end{definition}
    \begin{definition}[非齐次边值条件唯一解存在充要条件]
            对于(3.22)的齐次形式,$ \varphi_1(x),\varphi_2(x) $为其线性无关的解,记:
        \begin{align}
            \Delta=\begin{vmatrix}
                \varphi_1(a)&\varphi_2(a)\\\varphi_1(b)&\varphi_2(b)
            \end{vmatrix}
        \end{align}
        则非齐次边值问题唯一解存在充要条件是$\Delta \neq 0$

    \end{definition}
    \begin{definition}[齐次边值条件非零解存在充要条件]
        若二阶齐次线性微分方程满足齐次边值条件,则其非零解存在充要条件为:$ \Delta=0$
    \end{definition}
    \begin{definition}[齐次边界条件下的非齐次方程求解]
        格林函数.(remain to be done)
    \end{definition}
\chapter{常微分方程的基本理论}
    本章讨论微分方程初值问题解的存在性,唯一性,解的连续性,可微性.(由于作者数学证明能力极差,本章不作记录)
    \section{初值问题解的存在性与唯一性}
    \subsection{导数已解出的一阶方程的初值问题}
        本段研究导数已经解出的一阶微分方程,给定$t_0,x_0$,
        \begin{align}
            \frac{dx}{dt}&=f(t,x)\\
            x(t_0 )&=x_0
        \end{align}
        求方程(4.1)满足初值条件(4.2)的解$x = x(t_0)$的问题,为初值问题.\\
        \begin{definition}[lipschitz条件与解]
            对于(4.1)与(4.2),假设函数$f(t,x)$在闭的矩形区域:\\\centering
            \begin{aligned}
                \bar{R}:\quad|t-t_0|<a,|x-x_0|<b
            \end{aligned}
            \\\raggedright
            是连续的,并且关于x满足lipschitz条件,即:存在正的常数$N>0$,使得当$ (t,x_1)\in \bar{R},(t,x_2)\in \bar{R}$时,不等式成立:
            \\\begin{align}
                \left|f(t,x_1)-f(t,x_2)\right|\leq N\left|x_1-x_2\right|
            \end{align}
            则在区间$ [t_0-h,t_0+h]$上存在唯一的连续可微函数适合方程(4.1)与(4.2).\\
            其中,\begin{align}
                h&=\min(a,\frac bm)\\
                M&=\max_{(t,x)\in\boldsymbol{R}}|f(t,x)|
            \end{align}
        \end{definition}
\chapter{定性理论初步}
    本章介绍定性理论和稳定性理论的初步知识,i.e.相平面/奇点/极限圈的基础知识,解的稳定性概念和稳定性理论的基本定理.\\
    \section{相平面与奇点}
    \subsection{相平面}
        \begin{definition}[自治系统,相平面,相轨线]
            右端不显含自变量t的微分方程\\\centering
            \begin{equation}
                \begin{aligned}
                    \begin{cases}
                        \frac{dx}{dt} &= f(x,y)\\
                        \frac{dy}{dt} &= g(x,y)
                    \end{cases}
                \end{aligned}
            \end{equation}\\\raggedright
        成为自治系统的微分方程,简称自治系统.其中,$f(x,y),g(x,y)$连续可微.\\
        想象三维坐标系中,以时间t为观测顺序,考察坐标点(t,x,y)的变化规律.此即为微分方程组的积分曲线,其在(x,y)平面的投影,即为方程组的相轨线.此平面称之为相平面\\
        \end{definition}
        微分方程组的积分曲线不相交,而相轨线要么相交,要么重合.
        \begin{definition}[平衡点/奇点]
            设(5.1)的解为$ x= \varphi(t), y= \Phi(t)$,当两个解$\boldsymbol{\varphi}(t)\equiv\boldsymbol{a},\boldsymbol{\psi}(t)\equiv\boldsymbol{b}$时,如果(5.1)右边为0,则相轨线退化为点.称(a,b)为自治系统(5.1)的奇点,或平衡位置.\\
            对(5.1)作线性变换,将奇点移动至(0,0),考虑到(5.1)已是连续可微的,则方程组在(0,0)可以表示为:
            \begin{equation}
                \begin{aligned}
                    \begin{cases}
                        \frac{dx}{dt} &= \alpha x + \beta y + X(x,y)\\
                        \frac{dy}{dt} &= \gamma x + \delta y + Y(x,y)
                    \end{cases}
                \end{aligned}
            \end{equation}
            其中,$X(x,y),Y(x,y)$为$\sqrt{x^2+y^2}$的无穷小.
        \end{definition}
    \subsection{二阶线性系统的奇点}
    本节讨论(5.2)在奇点(0,0)周围的相轨线形状.思路是:讨论一次近似方程组的相轨线的形状.\\
    \begin{equation}
        \begin{aligned}
            \begin{cases}
                \frac{dx}{dt} &= \alpha x + \beta y \\
                \frac{dy}{dt} &= \gamma x + \delta y
            \end{cases}
        \end{aligned}
    \end{equation}
    该方程组的系数矩阵为:
    \begin{align}
        A=\begin{pmatrix}
            \alpha&\beta\\\gamma&\delta
        \end{pmatrix}
    \end{align}
    当$ \det(A)\neq 0$时,方程组(5.3)有唯一零点(0,0).当行列式不等于0时,需根据参数为0与否,判断奇点所在的空间,可能是直线,可能是全平面.\\
    一方面可根据特征值判断解的形式,通过写出比值极限,判断相轨线形状.\\
    \begin{enumerate}
        \item[1] $\lambda_1<\lambda_2<0$\\当$t\rightarrow-\infty$时,相轨线趋于无穷远,即随着时间前进,相轨线趋于(0,0),奇点为稳定结点.
        \item[2] $0<\lambda_1<\lambda_2$\\相轨线随时间趋于无穷远.奇点为不稳定结点.
        \item[3] $\lambda_1<0<\lambda_2$\\相轨线双向趋离,奇点为鞍点.
    \end{enumerate}
    另一方面,(5.3)二式作比值,得到相轨线斜率方程:
    \begin{align}
        \beta k^2+(\alpha-\delta)k-\gamma=0
    \end{align}
    以上叙述了特征值不同的情况,当特征值相同时,由于重根影响特征向量的求解,因而对于二阶自治系统,有两种情况:
    \begin{enumerate}
        \item[1]$A$相对于$\lambda$有两个线性无关的特征向量,则:\\\centering
                \begin{aligned}
                    x=(c_1h_{11}+c_2h_{12})e^{\lambda t}\\y = (c_1h_{21}+c_2h_{22})e^{\lambda t}
                \end{aligned}
         \\\raggedright 为其解,特征值小于零,称奇点为稳定的临界结点,反之不稳定的临界结点.
        \item[2]当需要以循环向量表示特征向量时,情况与[1]相似.
    \end{enumerate}
    临界结点与退化结点:此二者的区别,当斜率方程(5.5)只有1根时,为退化结点;当其为不定方程时(即矩阵为对角阵且对角线元素相同时),为临界结点.\\
    最后,当特征值为共轭复根$ \lambda=\mu \pm i\nu $时,相轨线方程为:
    \begin{align}
        \tilde{r}=\tilde{r}_{0}\exp\left(-\frac{\mu}{\nu}\left(\theta-\theta_{0}\right)\right)
    \end{align}
    若$\mu<0$,相轨线是环绕原点的螺旋线,称之为稳定焦点;$\mu>0$,为不稳定焦点.当$\mu=0$,相轨线为同心椭圆,称奇点为中心.\\
    \begin{theorem}[非线性系统的Perron定理]
        A的特征值实部不为0且非线性部分无穷小($ X^2+Y^2 =o{(x^2+y^2)^{k,k>1}}$)时,自治系统的相轨线由其一次近似决定.
    \end{theorem}
    \begin{example}[有阻尼单摆运动方程(非线性)]
        \begin{align}
            \frac{d^2\varphi}{dt^2}+b\frac{d\varphi}{dt}+\frac gl\sin\varphi=\mathbf{0}
        \end{align}
        置$x=\varphi,y=\frac{d\varphi}{dt}$,则:
        \begin{align}
            \begin{cases}
                \frac{dx}{dt}&=y\\ \frac{dy}{dt}&=-\frac gl \sin x-by
            \end{cases}
        \end{align}
        其近似方程(等价)的特征方程为:$\lambda^2+b\lambda+\frac gl=0$\\
        $\frac gl$为正数,阻尼系数为正数.当b足够小时,根为共轭复数,实部<0,为稳定焦点.\\
        现在讨论$(\pi,0)$,作平移变换后,特征方程常数项变号,有2不等实根,为鞍点.\\
        综上所述,对于单摆运动,当初始条件较小时,会作小角度的简谐运动;最上端虽然满足方程组,但是为鞍点;当初始条件较大时,转过若干圈后,会再次作小角度的简谐运动.\\


    \end{example}
    \section{极限圈}
    本节研究二阶非线性自治系统.孤立的周期解(闭轨线)称为极限环,即极限环邻域内无其他闭轨.
        \begin{align}
            \begin{cases}
                \frac{dx}{dt}=f(x,y)\\\frac{dy}{dt}=g(x,y)
            \end{cases}
        \end{align}
        \subsection{闭轨线存在的判别法}
        \begin{definition}[Bendixson否定判据]
            设$f,g$在单连通区域$D$内连续可微,且$\frac{\partial f(x,y)}{\partial x}+\frac{\partial g(x,y)}{\partial y}\neq =0$则系统(5.9)在D中不存在闭轨线.
            证明可通过格林公式,反证法得到.
        \end{definition}
    \section{解的稳定性}
    上述讨论了奇点的稳定性.现在给出一阶常微分方程组的解的稳定性的定义与讨论方法.\\
    一阶常微分方程组为:
    \begin{align}
        \frac{dx_k}{dt}=f_k(t,x_1,x_2,\cdots,x_n),(k=1,2,\cdots,n)
    \end{align}
    矩阵形式为:
    \begin{align}
        \frac{d\mathbf{\mathit{x}}}{dt}={\mathbf{\mathit{f}}}(t,{\mathbf{\mathit{x}}})
    \end{align}
    当${\mathbf{\mathit{x}}}={\mathbf{\mathit{a}}}$时,若有${\mathbf{\mathit{f}}}( t,{\mathbf{\mathit{a}}})={\mathbf{\mathit{0}}}$,
    则称${\mathbf{\mathit{x}}}={\mathbf{\mathit{a}}}$为方程组的解,或平衡位置.当{\mathbf{\mathit{f}}}与t无关时,该系统称为自治系统.平衡位置称为奇点.
    \subsection{李雅普诺夫稳定性定义}
\end{document}
